%\documentstyle[epsf,twocolumn]{jarticle}       %LaTeX2.09仕様
\documentclass[twocolumn]{jarticle}     %pLaTeX2e仕様
%\documentclass[dvipdfmx,autodetect-engine]{ujarticle}	%???
%%%%%%%%%%%%%%%%%%%%%%%%%%%%%%%%%%%%%%%%%%%%%%%%%%%%%%%%%%%%%%
%%
%%  基本 バージョン
%%
%%%%%%%%%%%%%%%%%%%%%%%%%%%%%%%%%%%%%%%%%%%%%%%%%%%%%%%%%%%%%%%%
\setlength{\topmargin}{-45pt}
%\setlength{\oddsidemargin}{0cm}
\setlength{\oddsidemargin}{-7.5mm}
%\setlength{\evensidemargin}{0cm}
\setlength{\textheight}{24.1cm}
%setlength{\textheight}{25cm}
\setlength{\textwidth}{17.4cm}
%\setlength{\textwidth}{172mm}
\setlength{\columnsep}{11mm}

\kanjiskip=.07zw plus.5pt minus.5pt		%漢字と漢字の間に小さなグルーが入っている?その設定らしい


%【節がかわるごとに(1.1)(1.2) …(2.1)(2.2)と数式番号をつけるとき】tex
%\makeatletter
%\renewcommand{\theequation}{%
%\thesection.\arabic{equation}} %\@addtoreset{equation}{section}
%\makeatother

%\renewcommand{\arraystretch}{0.95} 行間の設定

%%%%%%%%%%%%%%%%%%%%%%%%%%%%%%%%%%%%%%%%%%%%%%%%%%%%%%%%
\usepackage[dvipdfm]{graphicx}   %pLaTeX2e仕様(要\documentstyle ->\documentclass)
\usepackage{url}
%%%%%%%%%%%%%%%%%%%%%%%%%%%%%%%%%%%%%%%%%%%%%%%%%%%%%%%%

\begin{document}

\twocolumn[		%全体を二段表示する場合には,一段表示したい部分を\twocolumn[]で囲めば良い
\noindent
\hspace{1em}

令和 5 年 1 月 24 日 (火) 情報工学実験 II 発表資料
\hfill
\ \ B3 味岡 陽紀

\vspace{2mm}
\hrule		%これはタイトルを囲む横線
\begin{center}
{\Large \bf Box Embedding による単語の分散表現獲得手法の検証}
\end{center}
\hrule
\vspace{3mm}
]

\section{はじめに}
深層学習など機械学習による自然言語処理は盛んな研究分野の一つである. 
自然言語処理はもちろん機械学習においても単語の表現や処理方法は文章の意味を計算によって解析するために必要かつ根幹的な要素である.
近年, Word2Vec のような単語の分散表現ベクトルが広く使われる一方で単語の意味で表現しきれない要素があることが認識されている. 
その代表例として点でしかデータを表すことができないというものがある. 
単語の埋め込みにおいてこのことは単語の意味の包含関係や階層関係といった集合的性質を自然に表現できないということに起因する. 
このような問題を解決するための表現として領域表現が提案されている. 
本実験では始点と終点の組で表される「箱」への単語埋め込み表現である Box Embedding を扱う. 

Box Embedding の獲得手法の一つである Word2Box による単語埋め込み表現を獲得し, その埋め込み表現が単語間の意味をどのように表しているかを実験により確認した. 

\section{要素技術}
\subsection{Box Embedding}
Box Embedding は d 次元のベクトルの組からなる「箱」の領域で単語を表現するものであり, Vilnis Luke ら\cite{box-lattice}によって提案された. 
図 \ref{fig:box_embedding2d} に Box Embedding の例を示す. 
図 \ref{fig:box_embedding2d} では mammal と carnivore の「箱」の共有部分にて cat の「箱」が存在するように, Box Embedding では「箱」同士の重なりによって単語の意味の包含関係や階層関係を表現する. 

Box Embedding は「箱」の重なりを調整することで表現を獲得する. しかし, Vilnis Luke らによって提案された方法では「箱」が重ならないと勾配が計算できず最適化が難しくなる問題がある.
この問題に対して「箱」の境界を滑らかだとして領域を計算することで「箱」同士が離れていても勾配を求められ最適化できる. 
図 \ref{fig:gumbel_box} にGumbel 分布を適用した 2 次元での「箱」を示す. 
Gumbel 分布を適用することで「箱」をうまく最適化することができ Box Embedding を獲得しやすくなる \cite{gumbel-box}. 

\begin{figure}[t]
  \centering
  \caption{2 次元での Box Embedding の例 (文献 \cite{li2018smoothing} Figure 1 引用)}
  \label{fig:box_embedding2d}
  \includegraphics[width=70mm]{BoxEmbedding_2d.png}
\end{figure}

\begin{figure}[t]
  \centering
  \caption{2 次元での Gumbel Box の例 (文献 \cite{pmlr-v161-boratko21a} Figure 3 引用)}
  \label{fig:gumbel_box}
  \includegraphics[width=70mm]{GumbelBoxes.png}
\end{figure}

% \subsection{Word2Vec}
% Word2Vec \cite{word2vec} は 2013 年に Google の Tomas Mikolov らによって発表された単語のベクトル埋め込み表現の生成モデルである. 
% Word2Vec は 2 層ニューラルネットワークであり, 対象とする単語の周辺に現れる単語の頻度をもとに類似する周辺単語を持つ単語とのベクトルの類似度が大きくなるように学習する. 
% 周辺の単語から対象の単語を予測し学習する CBOW モデルと対象の単語から周辺の単語を予測し学習する Skip-gram モデルがある.

\subsection{Word2Box}
Word2Box \cite{dasgupta-etal-2022-word2box} は Sankar Dasgupta らによって提案された手法であり, 単語の領域表現の Box Embedding を教師なし学習で獲得するものである. 
学習方法は Word2Vec \cite{word2vec} と同様で, 似た意味の単語は似た文脈に現れるという仮説をもとに対象単語とその周辺単語の関係を学習することで獲得する. 
また, 今回の実験では CBOW モデルのように対象の単語とその周辺単語との「箱」同士の重なりが大きくなるように学習する. 
加えて, 「箱」をただ大きくする最適化にならないように適当な語を負例として抽出し周辺単語との重なりを小さくなるように学習もする. 

\section{データセット}
\subsection{Penn Treebank データセット}
Penn Treebank データセット \cite{ptb} は 	Mitchell P. Marcus らによって作成された 450 万語を超える大規模英語コーパスであり, コーパスには品詞情報も付加されている. 
この実験では数字を全て `N' , 語彙に含まれない単語を ``\textless unk\textgreater'' , パディングを ``\textless pad\textgreater'' としてそれぞれ一つの単語と見なして語彙数が 1 万語になるように調整されたものを用いた.

\subsection{単語類似度データセット}
実験で単語の類似性を確かめるために単語類似度データセットとして Simlex-999 \cite{simlex999} と WordSim-353 \cite{wordsim353}, YP-130 \cite{yp-130}, MEN \cite{men}, MC-30 \cite{mc-30}, RG-65 \cite{rg-65}, VERB-143 \cite{verb-143}, Stanford RW \cite{stanford-rw}, Mturk-287 \cite{mrutk-287}, Mturk-771 \cite{mturk-771}, SimVerb-3500 \cite{gerz-etal-2016-simverb} を用いた. 
ただし, 各データセットの特徴に関しての詳細はそれぞれの文献を参照してもらいたい. 
また, WordSim-353 は関連する単語組で構成されるサブセットである WS-353 (Rel) と類似する単語組で構成されるサブセットである WS-353 (Sim) も用いた. 

\section{実験}
Word2Box の著者による実装\footnote{\url{https://github.com/iesl/word2box}}を用いて Penn Treebank コーパスから単語の埋め込みモデルを獲得をした. 
表 \ref{tb:model_parameter} にモデルの学習パラメータを示す. 
各単語類似度データセットをもとに類似度の高い単語の組について埋め込み表現同士での類似度を計算し, それぞれの単語においてもう一方の単語が類似度で学習した語彙の中で何番目に位置するかを求めた. 
各類似度データセットは数値の基準がそれぞれ異なるため, 基準値を上回る単語組を実験に用いた. 
表 \ref{tb:dataset_detail} に各データセットに対して定めた基準値を示す. 
ただし, 基準値は各データセットの類似度の最大値の 80 \% としている. 
求めた結果から Word2Box が 2 単語間の類似性を適切に学習し Box Embedding で表現できているかを確認した. 

\begin{table}[th]
  \centering
  \caption{Word2Box モデルの学習パラメータ}
  \label{tb:model_parameter}
  \begin{tabular}{c|c} \hline
      parameter&value \\ \hline
      次元数&64 \\
      バッチサイズ&4096 \\ 
      エポック数&10 \\
      ウィンドウサイズ&5 \\
      ネガティブサンプル数&2 \\
      サブサンプル閾値&0.001 \\
      学習率&0.004204091643267762 \\
      \hline
  \end{tabular}
\end{table}

\begin{table*}[th]
  \centering
  \caption{類似度データセットから抽出した単語組}
  \label{tb:dataset_detail}
  \scalebox{0.63}[0.63]{
		\begin{tabular}{c|ccccccccccccc}
			\hline
			
      & Stanford RW & RG-65 & YP-130 & MEN & MC-30 & Mturk-287 & SimVerb-3500 & SimLex-999 & Mturk-771 & WS-353(Sim) & WS-353(All) & WS-353(Rel) & VERB-143 \\ \hline
			基準値  & 8.0         & 3.2   & 3.2    & 40  & 3.2   & 4.0       & 3.2          & 8.0        & 4.0       & 8.0         & 8.0         & 8.0         & 3.2      \\
			該当組数 & 36          & 4     & 12     & 166 & 3     & 20        & 1053         & 68         & 85        & 16          & 31          & 14          & 0        \\ \hline
			\end{tabular}
  }
\end{table*}

\section{結果と考察}
表 \ref{tb:result} に実験結果を示す. 
どのデータセットにおいても類似度が 1 から 99 番目に含まれる類似単語のペアはほとんど存在せず, 上位 1000 番目まででも含まれる類似単語ペアの割合は少数となった. 
% また, 表 \ref{tb:rankdiff} のようにデータセットの種類にかかわらずほとんどの単語ペアにおいて類似度の順位を求める対象単語を入れ替えると順位が大きく変動することが確認された. 
% ただし, ここで大きく変動したとする基準は順位が 1000 異なる場合とした. 
% この大きく変動する組においても単語組が類似性を示す結果とはならなかった. 

続いて表 \ref{tb:sim_words} に類似度の順位が 1 から 99 番目に含まれていた単語組を示す. 
含まれていた単語組と含まれなかった単語組との間に使われる文脈や意味において法則性を見出すことはできなかった. 
また, データセット間での特徴や法則性も見受けられなかった. 

\begin{table}[th]
  \caption{類似単語ペアが含まれる順位}
  \label{tb:result}
  \begin{tabular}{c|ccc}
    \hline
    順位           & 1-99 & 100-499 & 500-999 \\ \hline \hline
    Stanford RW  & 0    & 7       & 0       \\
    RG-65        & 0    & 0       & 0       \\
    YP-130       & 0    & 2       & 2       \\
    MEN          & 6    & 24      & 19      \\
    MC-30        & 0    & 0       & 0       \\
    Mturk-287    & 0    & 5       & 0       \\
    SimVerb-3500 & 19   & 74      & 86      \\
    SimLex-999   & 5    & 8       & 9       \\
    Mturk-771    & 1    & 11      & 3       \\
    WS-353(Sim)  & 1    & 1       & 2       \\
    WS-353(All)  & 2    & 1       & 2       \\
    WS-353(Rel)  & 1    & 0       & 0       \\
    VERB-143     & 0    & 0       & 0       \\ \hline
  \end{tabular}
\end{table}

% \begin{table}[th]
%   \centering
%   \caption{対象単語入れ替えによる順位変動}
%   \label{tb:rankdiff}
%   \begin{tabular}{c|c|c}
%     \hline
%     データセット       & 変動が大きい組 & 組の総数 \\ \hline \hline
%     Stanford RW  & 28      & 36   \\
%     RG-65        & 4       & 4    \\
%     YP-130       & 9       & 12   \\
%     MEN          & 123     & 166  \\
%     MC-30        & 3       & 3    \\
%     Mturk-287    & 14      & 20   \\
%     SimVerb-3500 & 847     & 1053 \\
%     SimLex-999   & 54      & 68   \\
%     Mturk-771    & 53      & 85   \\
%     WS-353(Sim)  & 14      & 16   \\
%     WS-353(All)  & 24      & 31   \\
%     WS-353(Rel)  & 10      & 14   \\
%     VERB-143     & 0       & 0    \\ \hline
%   \end{tabular}
% \end{table}

\begin{table*}[th]
  \centering
  \caption{データセットごとのモデルによって類似度が高いとされた単語組}
  \label{tb:sim_words}
  \begin{tabular}{c|c}
    \hline
    データセット       & 類似単語組    \\ \hline \hline
    MEN          & (flight, plane), (highway, traffic), (cold, frozen), (mountain valley), (bay, beach), (burger, meat)  \\ \hline
    SimVerb-3500 & \begin{tabular}[c]{@{}c@{}}(let, release), (tell, testify), (release, go), (express, tell), (save, rescue), (release, relax), (tell, predict),\\  (spend, use), (say, tell), (tell, respond), (fight, defend), (rescue, help), (leave, release), (release, drain),\\  (release, loosen), (remove, release), (rescue, find), (gamble, bet), (gamble, risk)\end{tabular} \\ \hline
    SimLex-999   & (essential, necessary), (area, zone), (pact, agreement), (physician, doctor), (expand, grow)  \\ \hline
    Mturk-771    & (season, winter) \\ \hline
    WS-353(All)  & (king, queen), (opec, oil) \\ \hline
    WS-353(Rel)  & (opec, oil)    \\ \hline
    WS-353(Sim)  & (king, queen)   \\ \hline
  \end{tabular}
\end{table*}

\section{今後の課題}
今回の実験結果は Word2Box によって獲得した Box Embedding について意味が近い単語間では似た分散表現になるという性質を示しておらず, その点において十分な表現能力があることを確認できなかった. 
提案論文 \cite{dasgupta-etal-2022-word2box} において学習パラメータはいくつかの数値からランダムに選択することで決定し, 実験とは異なる学習データを用いているため, 学習パラメータの調整や学習用の文章の変更によって性能の向上の可能性を模索する余地があると考えられる. 
加えて今回は Word2Vec に代表されるようなベクトル埋め込み表現と Word2Box の性能差を確かめる実験ができなかったが他の領域埋め込み手法も加えて表現能力の比較検証の必要がある. 
また, 日本語のデータに対して Word2Box と Box Embedding を適用した場合においても意味の包含関係や階層関係が表現されるのかを確認したい. 

\bibliography{index}		%文献データベースindex.bibを使用する
\bibliographystyle{junsrt} 	%参考文献出力スタイル

\end{document}
