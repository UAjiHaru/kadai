%\documentstyle[epsf,twocolumn]{jarticle}       %LaTeX2.09仕様
\documentclass[twocolumn]{jarticle}     %pLaTeX2e仕様
%\documentclass[dvipdfmx,autodetect-engine]{ujarticle}	%???
%%%%%%%%%%%%%%%%%%%%%%%%%%%%%%%%%%%%%%%%%%%%%%%%%%%%%%%%%%%%%%
%%
%%  基本 バージョン
%%
%%%%%%%%%%%%%%%%%%%%%%%%%%%%%%%%%%%%%%%%%%%%%%%%%%%%%%%%%%%%%%%%
\setlength{\topmargin}{-45pt}
%\setlength{\oddsidemargin}{0cm}
\setlength{\oddsidemargin}{-7.5mm}
%\setlength{\evensidemargin}{0cm}
\setlength{\textheight}{24.1cm}
%setlength{\textheight}{25cm}
\setlength{\textwidth}{17.4cm}
%\setlength{\textwidth}{172mm}
\setlength{\columnsep}{11mm}

\kanjiskip=.07zw plus.5pt minus.5pt		%漢字と漢字の間に小さなグルーが入っている?その設定らしい


%【節がかわるごとに(1.1)(1.2) …(2.1)(2.2)と数式番号をつけるとき】tex
%\makeatletter
%\renewcommand{\theequation}{%
%\thesection.\arabic{equation}} %\@addtoreset{equation}{section}
%\makeatother

%\renewcommand{\arraystretch}{0.95} 行間の設定

%%%%%%%%%%%%%%%%%%%%%%%%%%%%%%%%%%%%%%%%%%%%%%%%%%%%%%%%
\usepackage[dvipdfm]{graphicx}   %pLaTeX2e仕様(要\documentstyle ->\documentclass)
\usepackage{url}
%%%%%%%%%%%%%%%%%%%%%%%%%%%%%%%%%%%%%%%%%%%%%%%%%%%%%%%%

\begin{document}

\twocolumn[		%全体を二段表示する場合には,一段表示したい部分を\twocolumn[]で囲めば良い
\noindent
\hspace{1em}

令和 5 年 月 日 () 情報工学実験 II 発表資料
\hfill
\ \ B3 味岡 陽紀

\vspace{2mm}
\hrule		%これはタイトルを囲む横線
\begin{center}
{\Large \bf Box Embedding による単語の分散表現獲得手法の検証}
\end{center}
\hrule
\vspace{3mm}
]

\section{はじめに}
機械学習や深層学習による自然言語処理が盛んである. 
自然言語処理において単語の表現や処理方法は文章の意味を計算によって解析するために必要不可欠な根幹的な要素である.
近年, Word2Vec のような単語のベクトル表現が広く使われる一方で単語の意味で表現することができない要素があることが認識されている. 
その代表的なものとして点でしかデータを表すことができないというものである. 単語のベクトル埋め込みにおいてこのことは単語の意味の包含関係や階層関係といった集合的な性質を自然に表現できないということを意味する. 
このような問題を解決するための表現として領域表現が提案されており, ガウス分布を用いたものや双曲空間を用いたものがある. 本実験では始点と終点の組で表される「箱」による埋め込み表現獲得手法 Box Embedding を単語に適用した Word2Box の単語表現と近い意味を持つ単語

このような領域表現のうちを単語へと適用したものが Word2Box である. 本実験では Word2Box での単語埋め込みによる表現と単語の概念の包含・階層関係の確認し, 
加えて Word2Vec との比較実験をした.

\section{要素技術}
\subsection{Box Embedding}
Box Embedding は d 次元のベクトルの組からなる「箱」の領域で単語を表現するものであり, (Box Lattice のやつを示す)で提案された.
Box Embedding は「箱」の重なりを調整することで表現を獲得するが, (Box Lattice の作者) らによって提案された方法では「箱」が重ならないと勾配が計算できず最適化が難しくなる問題がある.
この問題に対して, 「箱」の境界を滑らかだとして領域を計算することで「箱」同士が離れていても勾配を求められ最適化できる. 
この実験では「箱」の境界を滑らかにするために Gumbel 分布を用いる (Gambel Boxの作者) らによって提案された手法を用いる. 

\subsection{Word2Vec}
Word2Vec は 2013 年に Google の Tomas Mikolov らによって発表された単語のベクトル埋め込み表現の生成モデルである. 
Word2Vec は 2 層ニューラルネットワークであり, 対象とする単語の周辺に現れる単語の頻度をもとに類似する周辺単語を持つ単語とのベクトルの類似度が大きくなるように学習する. 
周辺の単語から対象の単語を予測し学習する CBOW モデルと対象の単語から周辺の単語を予測し学習する Skip-gram モデルがある.

\subsection{Word2Box}
Word2Box は Dasgupta らによって提案された手法であり, 単語の領域表現の Box Embedding を教師なし学習で獲得するものである. 
学習方法は Word2Vec の CBOW モデルと同様に対象の単語とその周辺単語の「箱」の重なりが大きくなるように学習する. 
「箱」の重なりは Gumbel 分布に基づいて

\section{データセット}
提案手法はうんたらかんたら〜

\section{実験}
実験をかくかくしかじか〜

\section{結果と考察}

\section{今後の課題}
今後の課題はうんぬんかんぬん〜


\bibliography{index}		%文献データベースindex.bibを使用する
\bibliographystyle{junsrt} 	%参考文献出力スタイル

\end{document}
